\documentclass[a4paper, conference]{IEEEtran}
%\documentclass[10pt,a4paper]{article}

% The amsmath and epsfig packages greatly simplify the process of adding
% equations and figures to the document, and thus their use is highly
% recommended.
% ------------


%\usepackage{pslatex} % -- times instead of computer modern
\usepackage[colorlinks=false,bookmarks=false]{hyperref}
\usepackage{booktabs}
\usepackage{graphicx}
\usepackage{subfigure}
\usepackage{xcolor}
\usepackage{multirow}
\usepackage{amssymb,amsmath}

\newcommand{\code}[1]{{\small{\texttt{#1}}}}
\usepackage{cite}

\renewcommand*\ttdefault{txtt}


\usepackage{listings}

\newcommand{\todo}[1]{{\emph{TODO: #1}}}

\newcommand{\martin}[1]{{\color{blue} Martin: #1}}
\newcommand{\abcdef}[1]{{\color{red} Author2: #1}}

% uncomment following for final submission
%\renewcommand{\todo}[1]{}
%\renewcommand{\author2}[1]{}
%\renewcommand{\martin}[1]{}

\begin{document}

\title{Notes and Comments on Chisel}

\author{Martin Schoeberl}

% Most conferences have their own commands for author headings.

%\author{\IEEEauthorblockN{Edgar Lakis, Martin Schoeberl}\\
%\IEEEauthorblockA{Department of Applied Mathematics and Computer Science\\
%Technical University of Denmark\\
%Email: \texttt{edgar.lakis@gmail.com}, \texttt{masca@imm.dtu.dk}}
%}


\maketitle \thispagestyle{empty}

\begin{abstract}
this paper is a collection of notes and comments on the Chisel HDL.
Maybe this will be useful for a future Chisel manual.
\end{abstract}


\section{Introduction}
\label{sec:intro}

\section{Comments}

What does tick() and what does step() do for testing?
  vars in step are input, the output is either set by the test bench to compare to, or set by the component output values when not set in the test bench.
How get to inner signals on testing?
What is the meaning if isTrace?

val x = expr is an assignment.
val y = type is a forward declaration
y := expr is then the assignment, := is called the wiring operator

The waveform shows the input of an register, not the output.
The waveform shows IO signals and registers.
The waveform shows only IO signals that are in use.

Try out multiple wiring operations, setup test play around with print of output, may be no -vcd now

This is the way to get a printable output:

\begin{verbatim}
      step(svars, ovars, false)
      val out = ovars(c.io.out).litValue()
\end{verbatim}


What is the meaning of data in Reg? Is it the input signal?

What does a step() when testing combinational circuits? Just change the input values, right? Clock is simply ignored.

Test bench is mix of Chisel and Scala code. The hardware description needs plain Chisel, 'no Scala constructs'. Which are the forbidden Scala constructs?

Hardware that is not really used is optimized away for the simulation.
This might be just a little bit inconvenient when signals cannot be seen in
the waveform. However, this is very annoying when it results in C code
that cannot be compiled.

1)What is a good way to monitor internal signals? for debugging

you can list the signals to monitor using the sequence args to Tester:

  class GCDTests(c: GCD) extends Tester(c, Array(c.io)) ...

where you just add extra nodes to the Array.  these will then appear in your ovars from calls to step:

Rewrite the above with the example talen from Patmos in chisel.

Take care that your components actually really extend Component. Otherwise
you get strange behavior.

\section{Notes from Talk with Jonathan}

Questions:

* Usage of Bundles for interface and register ok?

Yes, looks like. No direction when the Bundle is used for a register as well.

* How to write a table?

VHDL:
process(address) begin

\begin{verbatim}
case address is
    when "0000000000" => q <= "00000011000000000000000000000001";
    when "0000000001" => q <= "00000000011110111101000000010000";
    when "0000000010" => q <= "00000010010100100000000000110000";
    when "0000000011" => q <= "00000010110000000000010010000000";

One way in chisel:

  val v = Vec(4) { Bits(width=32) }
  v(0) = Bits("h_0002_00ff")  // maybe not executed
  v(1) = Bits("h_0004_0001")  // addi    r2 = r0, 1;
  v(2) = Bits("h_0006_0002")  // addi    r3 = r0, 2;
  v(3) = Bits("h_0208_2180") // add     r4 = r2, r3;

But better to use ROM:

  val muls = new Array[UFix](256)
  for (x <- 0 until 16; y <- 0 until 16)
    muls((x << 4) | y) = x * y
  val tbl = ROM(muls){ UFix(8) }

\end{verbatim}

* Asynchronous reset -- it will be considered. I might do a patch for that.

* Issue sometimes with dead code - not in the waveform and may
result in a C compiler error. But would be nice during development and debugging

* Maybe use Chisel source for the project? Only useful when doing Chisel development.

* Waveform display of a register is one off (shows the input and not the output):
will be considered.

\section{Conclusion}
\label{sec:conclusion}


\subsection*{Acknowledgment}

\todo{Sometimes we received some help. Sometimes external funding.}

%This work was partially funded under the
%European Union's 7th Framework Programme
%under grant agreement no. 288008:
%Time-predictable Multi-Core Architecture for Embedded
%Systems (\mbox{T-CREST}).

%\subsection*{Source Access}


% should be IEEE, but has urls - issue as usual :-(
%\bibliographystyle{IEEEtran}
\bibliographystyle{plain}
% Please do not add any references to msbib.bib.
% They get lost when I 'generate' is again (see Makefile)
\bibliography{msbib}

\end{document}
