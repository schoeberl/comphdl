
\documentclass[10pt, conference, compsocconf]{IEEEtran}


%\usepackage{pslatex} % -- times instead of computer modern

\usepackage{url}
\usepackage{booktabs}
\usepackage{graphicx}
%\usepackage{caption}
%\usepackage{subcaption}
\newcommand{\code}[1]{{\small{\textsf{#1}}}}
\newcommand{\codefoot}[1]{{\textsf{#1}}}
\usepackage{cite}
\usepackage{amsmath,amsthm}
\usepackage[usenames,dvipsnames]{xcolor}

\newcommand{\todo}[1]{{\emph{TODO: #1}}}

\usepackage{listings}
%\usepackage{subfigure}

\newcommand{\comment}[3]{\paragraph*{\textbf{#1}}{\color{#3}#2}}


\newcommand{\martin}[1]{\comment{Martin}{#1}{Blue}}


\begin{document}

\title{Working title: Compare HDLs}

\author{Martin}


\maketitle \thispagestyle{empty}

\begin{abstract}
Compare different HDLs.
\end{abstract}

\section{General}

Do some exploration of user base and available source code on all languages.

How many \emph{background} knowledge on programming language do we need for the new HDL?

\section{MyHDL}

\begin{itemize}
\item Very simple to download install
\item Python needs to be learned, some VHDL/Verilog knowledge is still needed
\end{itemize}

Installed on Python 2.7. Does it need a 2.x Python or would it run on 3.x as well?

How large is the user base on MyHDL? Any open-source projects?

see \url{http://thread.gmane.org/gmane.comp.python.myhdl/2701}

\section{Notes (from iPad/Dropbox)}

MyHDL

Having min and max values that cannot be implemented so in HW is strange. I would prefer just signed/unsigned with a power of 2 range.

MyHDL is not a new language, but. python package. Python is the language. If a HW description is embedded in a full blown general purpose language with a lot of libraries, how easy is it to see the boundaries of what is synthesizable hardware, what is test benches, unit tests, and even generator code?

There is no indication is parameters are input or output signals.

Testing looks convenient with all the Python support.

Are signal records/structures possible?

General

Is tri state supported?

Shall I look into C based HDLs?

SystemC?

SystemVerilog

\section{Conclusion}
\label{sec:conclusion}



\bibliographystyle{IEEEtran}
%\bibliographystyle{abbrv} % similar to IEEE without URLs
% pleas add bibs into other.bib as msbib.bib is 'generated'
\bibliography{msbib}

\end{document}

